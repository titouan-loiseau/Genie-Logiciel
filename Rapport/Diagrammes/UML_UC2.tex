\documentclass{standalone}
\usepackage{tikz}
\usepackage{tikz-uml}

\begin{document}
    \begin{tikzpicture}
        \begin{umlpackage}[fill=white]{AppliBanque} % package-p
            \umlclass[fill=white]{DossierBancaire}{ % classe DossierBancaire
            }
            {
                + DossierBancaire() \\
                + get\_solde() : double \\
                + deposer(value: double) : void \\
                + retirer(value: double) : void \\
                + remunerer() : void
            }

            \umlclass[x=10,fill=white]{CompteBancaire}{ % classe CompteBancaire
            \# m\_solde : double
            }
            {
                + CompteBancaire() \\
                + get\_solde() : double \\
                + deposer(value: double) : void
            }

            \umlclass[x=10,y=-4,fill=white]{CompteEpargne}{ % classe CompteEpargne
            - taux : double
            }
            {
                + CompteEpargne(taux: double) \\
                + remunerer() : void
            }
            \umlinherit{CompteEpargne}{CompteBancaire}
            \umlunicompo[arg=m compte bancaire, mult=1, pos=0.5]{DossierBancaire}{CompteBancaire}
            \umlunicompo[geometry=|-,arg=m compte epargne, mult=1, pos=0.5]{DossierBancaire}{CompteEpargne}
        \end{umlpackage} 
    \end{tikzpicture}
\end{document}