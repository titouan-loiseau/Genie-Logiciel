\documentclass[tikz]{article}
\usepackage{tikz-uml}
\begin{document}
    \begin{tikzpicture}
        \begin{umlpackage}[fill=white]{AppliBanque} % package-p
            \umlclass[fill=white]{DossierBancaire}{ % classe DossierBancaire
            - taux\_remuneration : double
            }
            {
                + DossierBancaire() \\
                + getSolde() : double \\
                + depot(v: double) : void
            }

            \umlclass[x=8,fill=white]{CompteBancaire}{ % classe CompteBancaire
            - solde : double
            }
            {
                + CompteBancaire() \\
                + getSolde() : double \\
                + ajout(m: double) : void
            }

            \umlclass[x=8,y=-4,fill=white]{CompteEpargne}{ % classe CompteEpargne
            - taux : double
            }
            {
                + CompteBancaire(taux: double) \\
                + remunerer() : void
            }
            \umlinherit{CompteEpargne}{CompteBancaire}
            \umlunicompo[attr1=cb|1]{DossierBancaire}{CompteBancaire}
            \umlunicompo[geometry=|-,attr1=ce|1]{DossierBancaire}{CompteEpargne}
        \end{umlpackage} 
    \end{tikzpicture}
\end{document}